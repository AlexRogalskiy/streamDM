\documentclass[a4paper,11pt]{article}
\usepackage[utf8]{inputenc}
\usepackage{palatino}
\usepackage{xspace}
\usepackage{fullpage}
\usepackage{microtype}
\usepackage{hyperref}

%Macros
\def\streamdm{\textsc{streamDM}\xspace}

%opening
\title{\streamdm: Advanced Data Stream Mining with Spark Streaming}
\author{Albert Bifet, Silviu Maniu, Jianfeng Qian, Guangjian Tian\\
%Alphabetical Order
Huawei Noah's Ark Lab \\
Science Park, Sha Tin, Hong Kong
}
\date{}

\begin{document}

\maketitle



Real-time Analytics are becoming increasingly important due to the large amount
of data that is being created continuously. Drawing from our experiences at
Huawei Noah's Ark Lab, we present {\bf \streamdm}, a new open source data mining and
machine learning library, designed on top of Spark Streaming, an extension of
the core Spark API that enables scalable, high-throughput, fault-tolerant stream
processing of data streams.  \streamdm is designed to be easily extended and
used, and is the first library to contain advanced stream mining algorithms for
Spark Streaming. 
 
Due to the large amount of data that is created -- and needs to
be processed -- in real-time streams, methods on such streams need to be extremely
time-efficient while using very small amounts memory. \streamdm
includes advanced stream mining algorithms, and is intended to be the gathering
point of practical implementation and deployments for large-scale data
streams.

This new library will contain methods for classification, regression, clustering
and frequent pattern mining. In its current iteration, it contains Stochastic
Gradient Descent, Perceptron, Naive Bayes for classification, and CluStream for
clustering streams. Compared to the current main machine learning libraries on
top of Spark (\emph{spark.ml} and \emph{MLlib}), the main advantages of
\streamdm are:
\begin{itemize}
  \item {\bf Ease of use} Experiments can be executed from the command-line, without
   the need for compiling tasks, similar to the WEKA and MOA libraries.
 \item {\bf No dependence on third-party libraries} For example, MLlib uses the
   linear algebra package Breeze, which in turn depends on netlib-java (among
   others).  Due to licensing issues, \emph{netlib-java}'s native libraries are
   not included in MLlib’s dependency set under default settings, complicating
   the compilation process. We aim to avoid this in \streamdm. 
 \item {\bf Ease of extensibility}, relying on a simple but powerful hierarchy of
   classes.
 \item {\bf Advanced stream machine learning methods} such as streaming decision trees,
   streaming Random Forests, streaming clustering methods as CluStream and
   StreamKM++. To the best of our knowledge, this is the first planned
   implementation of such methods on top of Spark Streaming.
\end{itemize}

In terms of future implementation, we expect to include streaming decision
trees, streaming Random Forests and StreamKM++ in the next release, planned for
late June 2015.

The current version of \streamdm is available at
\url{http://streamdm.noahlab.com.hk/} and \url{https://mloss.org/software/view/???/}


\end{document}
