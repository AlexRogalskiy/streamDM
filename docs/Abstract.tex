\documentclass[a4paper,10pt]{article}
\usepackage[utf8]{inputenc}
\usepackage{palatino}
\usepackage{xspace}
\usepackage{hyperref}

%Macros
\def\streamdm{\textsc{streamDM}\xspace}

%opening
\title{\streamdm: Advanced Data Stream Mining with Spark Streaming}
\author{Albert Bifet, Silviu Maniu, Jianfeng Qian, Guangjian Tian\\
%Alphabetical Order
Huawei Noah's Ark Lab \\
Hong Kong
}

\begin{document}

\maketitle

\begin{abstract}


Real-time Analytics are becoming increasingly important due to the large amount of data that is being created continuously. Drawing from our experiences at Huawei Noah’s Ark Lab, we present \streamdm, a new open source data mining and machine learning library, designed on top of Spark Streaming.
Spark Streaming is an extension of the core Spark API that enables scalable, high-throughput, fault-tolerant stream processing of data streams.
\streamdm is designed to be easily extended and used, and is the first library to contain advanced stream mining algorithms for Spark Streaming. 
 
The tools and algorithms in \streamdm are specifically designed for the data stream setting. Due to the large amount of data that is created – and needs to be processed – in real-time streams, such methods need to be extremely time-efficient while using very small amounts of time and memory. \streamdm includes advanced stream mining algorithms willing to be the gathering point of practical implementation and deployments for large-scale data streams.

This new library will contain methods for classification, regression, clustering and frequent pattern mining. Currently it contains Logistic Regression, Perceptron, Naive Bayes, and CluStream. The main advantages of \streamdm compared with spark.ml and MLLib will be the following:
\begin{itemize}
 \item Ease of use. Experiments can be executed from the command-line, as in WEKA or MOA.
 \item No dependece on third-part libraries, specially on the linear algebra package Breeze. MLlib uses  Breeze, which depends on netlib-java, and jblas that depend on native Fortran routines. Due to license issues, netlib-java’s native libraries are not included in MLlib’s dependency set under default settings. 
 \item Ease of extensibility
 \item Advanced machine learning methods as streaming decision trees, streaming Random Forests, streaming clustering methods as CluStream and StreamKM++. 
\end{itemize}

We expect to have the streaming decision trees, streaming Random Forests and StreamKM++ available in later June.

\streamdm is available from \url{https://mloss.org/software/view/???/}

\end{abstract}

\end{document}
